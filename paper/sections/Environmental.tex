\section{Environmental Simulation}
\label{sec:environmental}

To enhance ecological validity, the original recording synthesis procedure was modified to enable evaluation under two additional scenarios: recordings with additive noise and recordings in reverberant conditions. In the first scenario, nine test sets were prepared with different Signal-to-Noise Ratios (SNR) ranging from -10 to 60 dB, specifically at -10, -3, 0, 10, 20, 30, 40, 50, and 60 dB. This was achieved by adding decorrelated white noise signals to the binaural recordings originally used in the testing procedure.

In the reverberation scenario, six different rooms were simulated with reverberation times ranging from 0.1 to 3 s, measured using the RT60 metric. The simulations were performed with MCRoomSim—a multichannel `shoebox' room acoustic simulator based on image source and diffuse rain algorithms implemented as a MATLAB package \cite{wabnitz_room_2010}. This simulator enabled the creation of reverberation simulations used to generate Binaural Room Impulse Responses (BRIRs) based on provided HRTFs, with the number of virtual speakers matching the~spatial density of measurement points in the~HRTF database. The virtual listener, modeled as a head with two receivers representing ears, was positioned in the~center of the room. The receivers were configured to filter the input signal directionally using head-related impulse responses from the given HRTF database. The distance between each virtual impulse source and the head center matched the measurement radius of the given HRTF database, ranging from 0.9 to 1.95~m. The room reverberation characteristics were controlled by configuring the following parameters: room width and depth (2--5 m), height (2.5--5 m), wall absorption coefficients (0.05--0.95), and wall scattering coefficients (0.01--0.8).
