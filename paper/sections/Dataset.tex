\section{Dataset Preparation}

The experimental evaluation was conducted using a corpus of 23,040 synthesized binaural music recordings. The source material comprised 192 publicly-available multi-track recordings spanning diverse musical genres including rock, jazz, pop, and classical music. The number of tracks ranged from 5 to 62, with a median of 9.

To ensure robust evaluation across diverse Head-Related Transfer Function (HRTF) characteristics, the synthesis process incorporated 30 HRTF databases (see Table \ref{table:hrtfs} in Appendix for a detailed list). These databases were evenly divided into measurements from human subjects (15 databases) and measurements from artificial heads (15 databases), including industry-standard devices such as the Neumann KU 100 and KEMAR DB-4004. Distances between the head and loudspeaker during HRTF measurements ranged from 0.9 to 1.95 meters, with a median of 1.2 meters.

For each combination of multi-track recording and HRTF database, four unique binaural versions were synthesized by randomly varying two ensemble parameters: location ($\phi$) and width ($\omega$). Within these spatial constraints, individual tracks in each recording were randomly assigned to specific source positions ($\theta_i$). Prior to synthesis, all tracks were loudness-normalized to -23~LKFS in accordance with ITU-R BS.1770-5 recommendations \cite{noauthor_itu-r_2023}, ensuring consistent relative levels across the corpus.

Multiple HRTFs and multi-track recordings were selected to create diverse binaural recordings, enhancing the model's generalisability. This diversity is crucial for HRTFs since the~specific HRTF used in real-world binaural synthesis is often unknown, making a single-HRTF model impractical for general use. Additionally, the large dataset provides essential training material for all machine learning models used in this study, with particular importance for the deep neural networks, as their performance benefits significantly from extensive data.

The binaural recordings were obtained using a binauralization procedure, implemented by convolving multi-track signals with head-related impulse responses from a specified HRTF database. The resulting binaural output signal, $y_c[n]$, for each stereo channel $c$ (left or right) at sample $n$ is given by the~following equation:
\begin{equation} \label{eq:1}
    y_c[n] = \sum_{i=1}^{N} \sum_{k=0}^{K-1} x_i[k] \times h_{c,\theta_i}[n-k],
\end{equation}
where $x_i$ denotes the signal of an individual sound source $i$ from the input music recording, and  $h_{c,\theta_i}$ represents the head-related impulse response for channel $c$ at location $\theta_i$ of source track $i$. Additionally, $N$ denotes the number of track sources in the~input multi-track recording, and $K$ represents the number of samples in the recording.

The synthesized recordings were truncated to 7 seconds following binauralization, with sine-squared fade-in and fade-out effects of 0.01 seconds applied. Subsequently, the signals were RMS-normalized, scaled by a factor of 0.9, DC-offset corrected, and stored as uncompressed files with a sample rate of 48 kHz and 32-bit resolution.

The binaural recordings were randomly split into training and test sets with a 2:1 ratio. To prevent information `leakage', this split was made in such a way that no multi-track recordings used for training were used for testing. To reduce the~complexity of the experiment, the HRTFs were shared between both sets, which could be seen as a limitation of this study. However, it is known that the human auditory system operates with HRTFs that undergo only minimal changes throughout life, mainly during infancy \cite{king_how_2001}. Therefore, this limitation could be considered consistent with how the human auditory system behaves in real life. 

The binauralization and split procedures implemented in this study are consistent with those originally described in the reference models \cite{antoniuk_blind_2023,antoniuk_ensemble_2024,antoniuk_estimating_2024}, with minor modifications. The~primary modification pertains to the spatial-spectrogram-based model, which utilized a single HRTF database and employed a~reduced parameter set. This modification had minimal impact on the results, as the method employs a deterministic approach rather than machine learning techniques, requiring the training set only for the optimization of two parameters.
